\documentclass{article}
\usepackage[utf8]{inputenc}
\usepackage{amsmath}

\title{Estructura TFG mates}
\renewcommand{\contentsname}{Índice de contenidos}

\begin{document}

\maketitle

\newpage
% Índice de contenido
\tableofcontents

\newpage

\begin{abstract}
    \begin{itemize}
        \item Qué vamos a estudiar en este trabajo (proponemos un modelo y realizamos simulaciones) y por qué es interesante (estudio contra enfermedades, sencillez pero eficacia,...)
    \end{itemize}
\end{abstract}

\section{Introducción}
    \subsection{Propósito del sistema inmune: respuesta eficaz ante patógenos.}
    
        \begin{itemize}
            \item Respuesta consolidada desde hace años por la evolución. Pretende mantenernos vivos el suficiente tiempo para reproducirnos.
            \item Este sistema no es invulnerable, se cometen fallos que dan lugar a enfermedades.
            \item No hay un órgano que regule los comportamientos de este sistema (al menos no se ha encontrado). La respuesta se basa en información local.
           
        \end{itemize}
    
    \subsection{Las células T y su comportamiento}
        \begin{itemize}
            \item  contar qué son las células T .Nos centraremos en el estudio de la dinámica de población de las células T.
            \item Número limitado de opciones predeterminadadas (como un software)
        \end{itemize}
        
\section{Reconocimiento de antígenos}
    Algo similar al apartado 2 del resumen.
    \subsection{Proceso de reconocimiento de antígenos}
    \subsection{Células naïve, efectoras y con memoria}
    \subsection{Clonal expansion y clonal contraction}
    
\section{Algoritmo de la decisión}
    
        
    \subsection{Otros estudios relacionados (?)}
        Debería incluir aquí aportaciones de otros trabajos de investigación. Otros puntos de vista. (algo breve) ?
        
    \subsection{Ciclo vital de la célula}
        Explicación del ciclo celular, de las fases que tiene y qué determina el paso de una fase a otra.
        \\
        Se mezcla con el siguiente punto
    \subsection{Suposiciones iniciales de nuestro modelo}
        Lo referente al punto 2.1 de Journal of Theoretical Biology
    \subsection{Ecuaciones}
        Ecuaciones (3), (4), (8).
        Dejo la (5) para otro apartado como en el resumen?
    
\section{Simulaciones}
    Aquí incluyo las simulaciones que haya hecho de las ecuaciones anteriores. Supongo que será algo similar al apartado 3 de Journal of Theoretical Biology.
    
\section{Conclusión}



\end{document}
