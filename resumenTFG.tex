

\documentclass{article}
\usepackage[utf8]{inputenc}

\title{¿Cómo nos defendemos de los patógenos? Las células T y su comportamiento}
\author{Belén Serrano Antón }


\begin{document}
	
	\begin{titlepage}
		\maketitle
	\end{titlepage}
	
	\section{Introducción}
	La habilidad de nuestro sistema inmune para combatir patógenos, ya sean internos (como un tumor) o externos (como una infección por microbios) es ciertamente apasionante. 
	\\
	\\
	Parece natural pensar que existe un ``órgano inmune" que controla a las células inmunes y les da información sobre cómo y cuándo actuar. Sin embargo, ese órgano, si existe, aún no se ha encontrado. Incita a pensar, por tanto, que las células inmunes basan su actuación en la información que encuentran a su alrededor. Y sobre esta suposición construiremos un modelo que describa las dos actuaciones básicas (división o muerte celular) que desarrollarán las células inmunes de nuestro estudio: las células T.
	
	\section{Reconocimiento de antígenos}
	 Como veníamos diciendo en la introducción, asumiremos que las células T toman sus decisiones basándose únicamente en información local, pero ¿cómo distinguen a los amigos de los enemigos?
	 \\
	 \\
	 Las células T están dotadas de un receptor de membrana, que en lo que sigue llamaremos TCR (T Cell Receptor), a través del cual son capaces de comunicarse con el exterior. Si el TCR no percibe ningún antígeno, la célula T quedará ``desactivada", en un estado naïve. Si, por el contrario, el TCR y un antígeno se encuentran se produce un proceso conocido como \textit{sinapsis inmmune} y la célula se activa, convirtiéndose en una \textit{célula efectora}, lista para combatir al antígeno, o una \textit{célula con memoria}, que guardará información sobre el patógeno por si volviera a aparecer en un futuro combatirlo más rápidamente. Es entonces cuando el número de células T comienza a aumentar (\textit{clonal expansion}). Y, una vez que el antígeno ha sido vencido, el número de estas células T decrece rápidamente (\textit{clonal contraction}), aunque con retardo respecto a la desaparición del antígeno.
	 
	 \section{Algoritmo de decisión}
	 Una vez que una célula T ha sido activada, esta puede tomar un número de opciones muy limitadas, las resumiremos en dos:
	 \begin{itemize}
	 	\item dividirse
	 	\item morir
	 	\item o diferenciarse en una restringida variedad de células T.
	 	\end{itemize}
	 Estas decisiones vendrán determinadas por la competición de dos moléculas inhibidoras: Retinoblastoma (Rb), que previene la expresión de genes necesarios para que la célula pueda continuar el ciclo celular y dividirse, y célula B linfoma-2 (Bcl-2), que bloqueará la muerte celular.
	 \\
	 Si la cantidad de Rb cae por debajo de cierto límite, la célula T comenzará la división celular. Por el contrario, si es la cantidad de Bcl-2 la que  cae por debajo de cierto límite, provocará la apóptosis de la célula T. 
	 \\
	 \\
	 Como ya habíamos aunuciado, la célula T se comunica con el exterior gracias a su TCR, y las variaciones en la cantidad de Rb y Bcl-2 dependerán de que sean fosforiladas por unas unas moléculas llamadas citoquinas.
	\\
	Ambos procesos, división y apóptosis, son excluyentes y, en cuanto uno de ellos comienza, es interrumpible, independientemente de la acción de las citoquinas. LAS CELULAS T CON MEMORIA NO TIENEN RECEPTORES DEATH
	\\
	\\
	Con toda esta información, ya estamos en condiciones de presentar las ecuaciones de nuestro modelo: 
	\begin{itemize}
	    \item Denotaremos por \textit{$c(t)$} y \textit{$a(t)$} la cantidad de Rb y Bcl-2 activa en tiempo t, respectivamente.
	    \item \textit{$R_{i}$} será el receptor de la i-ésima citoquina y \textit{$r_{i}(t)$} será la cantidad de ese receptor en tiempo t. 
	    \item $r_{T}$ es el número de señales TCR/antíeno  percibidas por la célula T correspondiente.
	\end{itemize} 
	 Presentamos las siguientes ecuaciones diferenciales:
	 
	 \begin{displaymath}
         \left\{ \begin{array}{l}
        \dot{c}(t) = \mu_{Tc}r_{T}(t) + \sum_{j=1}^{k}\mu_{jc}r_{j}(t)\\
        \dot{a}(t) = \mu_{Ta}r_{T}(t) + \sum_{j=1}^{k}\mu_{ja}r_{j}(t) \\
        \end{array}
        \right.
    \end{displaymath}
    Estableceremos que las condiciones $a(t)=0$ y $c(t)=0$ provocarán instantáneamente la transición desde la etapa de decisión a apóptosis o división celular, respectivamente.
	\\
	Si una célula muere, será eliminada de la población. De manera contraria, si esta se divide será sustituida por dos células del mismo tipo y cuyas condiciones iniales veremos más adelante.
	\\
	\\
	La ecuación que modelará el comportamiento del TCR vendrá dada por 

	\begin{displaymath}
        \begin{array}{ll}
        \dot{r}_{i}(t) = \lambda_{Ti}r_{T}(t) + \sum_{j=1}^{k}\lambda_{ji}r_{j}(t) & \mbox{para $i=1,...,k$} 
        \end{array}
    \end{displaymath}
	En ella expresamos el carácter lineal del TCR y su dependencia de la cantidad del resto de receptores y el número de señales percibidas por la célula T. 
	\\
	Las ecuaciones que hemos tomado son lineales porque suponemos que los TCR son independientes y tienen efectos acumulativos. Así mismo, nos permiten establecer que para configuraciones de membrana similares, las células T tomarán decisiones similares.
	\\
	\\
	Nos quedaría ahora establecer qué ocurre con los TCR cuando una célula se divide. 
	\\
	Al contrario que las células T desactivadas (naïve), las células T con memoria y las células T efectoras se dividen de manera simétrica. Es decir, comparten sus receptores de membrana con sus dos células hija siguiendo la ecuación:
	
	\begin{displaymath}
         \left\{ \begin{array}{l}
        r_{i0}^{1}= \delta_{i}^{x} r_{i}^{x}\\
        r_{i0}^{2}= (1-\delta_{i}^{x}) r_{i}^{x} \\
        \end{array}
        \right.
    \end{displaymath}
	Donde $\delta_{i}^{x}$ representa el ratio de receptores de membrana de tipo $R_{i}$ entre las células hijas, $r_{i0}^{1}$ y $r_{i0}^{2}$ denotan los valores iniciales de receptor $R_{i}$ en las células hijas 1 y 2, respectivamente, y  $r_{i}^{x}$ denota el número de receptores $R_{i}$ en la célula T $x$ en el momento de la división celular.
	
	\section{Conclusiones}
	Hemos visto un algoritmo muy simple, que modeliza las decisiones de las células T.
	\\
	A pesar de que el modelo se asemeja a lo que observamos experimentalmente, aún quedan muchas cosas por resolver, una de las más importantes es: ¿cómo puede ser que si el comportamiento de estas células depende de decisiones locales veamos un comportamiento global? Es decir, que las células T funcionen como un ``equipo" para protegernos de los antígenos. 
	
	
\end{document}