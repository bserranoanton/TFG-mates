\chapter{Conclusions and Future Work}
\label{cap:conclusions}

The studies collected in this document show the usefulness of mathematical methods in proposing models that not only reproduce observed facts but are also capable of suggesting explanations for them and make it possible to formulate predictions that can be verified, or discarded, experimentally. 


The models proposed in Chapters \ref{cap:descripcionTrabajo} and \ref{cap:modeloMacroscopico} are presented as possible explanations to a biological mechanism of great interest and only partially known, such as the population dynamics of T cells during an acute infection. Both models are well-founded, as their hypotheses are based on biological evidence. Thanks to these models we can reproduce and predict the behaviour of T cells during acute infection in different situations by varying the value of their parameters, without the need for expensive experiments and from two different points of view, the microscopic and the macroscopic. On the other hand, both are models open to the inclusion of new biological knowledge. 


From the microscopic model, we highlight its contrast to the hypothesis that the life of a T cell is determined by the \textit{antigenic stimulation} received during its activation. In this way, the cells generated in the \textit{clonal expansion} would have very limited control over their choice between division or apoptosis. However, the proposed model states that the encounters of a T cell with the \textit{antigen} are transmitted to the daughter cells using membrane receptors, which are distributed during cell division. This allows new cells to integrate this knowledge with their own experience with the \textit{antigen}, enabling cells that share the same ancestor to make different decisions. This shows that the heterogeneity of decisions observed during an immune response can be explained by a deterministic algorithm and can be independently executed by the T cells. The response given by T cells is specific to an \textit{antigen}, but not to a pathogen\footnote{A specific \textit{antigen} may be present in several pathogens, which may be very heterogeneous in terms of growth rates or escape mechanisms from the \textit{immune response}.}. This explains the fact that the mechanisms of pathogen recognition are not detailed in the model, giving it the capacity to adapt to different infection strategies \citep{JTB}


Meanwhile, the macroscopic model observes the population dynamics of T cells (\textit{clonal expansion} and \textit{contraction}) collectively. It, therefore, models their behaviour using second-order differential equations, based on two population properties: elasticity and inertia. This model yields interesting results. Specifically, it proposes that the mechanism of identification of the target population of T cells is determined by the growth rate of the T cell. In other words, those populations that grow rapidly are considered to be pathogenic, while those with reduced growth rates are tolerated. This proposal is compatible with the paradoxical fact that slow growth is the avoidance strategy of some tumour cells or viruses such as hepatitis C \citep{arias2015growth}.


Another question that we consider particularly relevant is the one raised in Chapter \ref{cap:redNeuronal}. In this chapter, we sought to establish a correspondence between the parameters of the two models seen in the previous chapters, since both show compatible population behaviour. The microscopic model is characterized by representing explicit biological characteristics of the cells by means of structural parameters, whose value remains fixed during the simulation. However, the meaning of the parameters of the macroscopic model, referring to the characteristics of inertia and elasticity of the T cell population, lacks a clear biological meaning. Nevertheless, one of the advantages of this model is the reduced number of parameters it has. Therefore, finding a parameter correspondence between the two models would be very useful, for example, to determine the parameters of the microscopic model that correspond to a certain \textit{immune response}. In this way, the response could be simulated using the macroscopic model, whose parameters are easier to adjust and subsequently establish the corresponding parameters in the microscopic model, which are easily interpreted. A first approximation to the resolution of this complex question is detailed in Chapter \ref{cap:redNeuronal}. The results obtained by the neural network are promising, but still insufficient to deduce a formal correspondence between both models. We hope to deepen this study in the near future.


