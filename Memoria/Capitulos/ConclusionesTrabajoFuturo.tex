\chapter{Conclusiones y Trabajo Futuro}
\label{cap:conclusiones}

%Conclusiones del trabajo y líneas de trabajo futuro.

%Antes de la entrega de actas de cada convocatoria, en el plazo que se indica en el calendario de los trabajos de fin de máster, el estudiante entregará en el Campus Virtual la versión final de la memoria en PDF. En la portada de la misma deberán figurar, como se ha señalado anteriormente, la convocatoria y la calificación obtenida. Asimismo, el estudiante también entregará todo el material que tenga concedido en préstamo a lo largo del curso.

Tras la lectura de este documento hemos podido comprobar que las matemáticas pueden convertirse en herramientas muy potentes también fuera de su ámbito más teórico, algo que aún las enriquece más. Concretamente, las colaboraciones con biólogos cada vez son más frecuentes y, a pesar de ser ciencias muy distintas, se han obtenido resultados relevantes como el que describíamos a lo largo de estas páginas. Aún son numerosas las preguntas que quedan por resolver en el campo de la biología, es precisamente en esos puntos donde las matemáticas, gracias a su poder de abstracción y simplificación, pueden formular modelos que no solo reproduzcan hechos observados sino que infieran nuevo conocimiento. Estas nuevas vías de investigación permiten, en muchos casos, avanzar en el estudio de una línea de pensamiento y descartar otras, dando cabida al progreso científico.

Los modelos propuestos en los Capítulos \ref{cap:descripcionTrabajo} y \ref{cap:modeloMacroscopico} se presentan como posibles explicaciones a un mecanismo biológico aún por determinar, como es la dinámica de población de las células T durante una infección aguda. Ambos modelos constituyen modelos robustos, flexibles y bien fundamentados, pues sus hipótesis las establecen evidencias biológicas. Gracias a estos modelos somos capaces de reproducir y predecir el comportamiento de las células T durante una infección aguda en distintas situaciones, sin necesidad de costosos experimentos ni un laboratorio, y desde dos puntos de vista diferentes, el microscópico y el macroscópico. Pero, sin duda, una de las características más importantes de estos modelos es que no son modelos cerrados, ambos están abiertos a la inclusión de nuevo conocimiento biológico. 

Otra de las cuestiones que aún busca respuesta es la planteada en el Capítulo \ref{cap:redNeuronal}. En este se buscaba establecer una correspondencia entre los parámetros de los dos modelos vistos en los capítulos anteriores. En este capítulo se detalla una primera aproximación para la resolución de este compleja cuestión. Los resultados obtenidos por la red neuronal no son despreciables pero aún insuficientes para poder deducir una correspondencia formal entre ambos modelos. Es por ello que queda a la espera de nuevo estudio e investigación.

