\chapter{Conclusiones y Trabajo Futuro}
\label{cap:conclusiones}

%Conclusiones del trabajo y líneas de trabajo futuro.

%Antes de la entrega de actas de cada convocatoria, en el plazo que se indica en el calendario de los trabajos de fin de máster, el estudiante entregará en el Campus Virtual la versión final de la memoria en PDF. En la portada de la misma deberán figurar, como se ha señalado anteriormente, la convocatoria y la calificación obtenida. Asimismo, el estudiante también entregará todo el material que tenga concedido en préstamo a lo largo del curso.

Como hemos podido observar a lo largo de este documento, son muchas las preguntas para las cuales la biología aún no tiene respuesta. Es por ello que las matemáticas toman un papel relevante como herramienta investigadora. En el caso que nos ocupa hemos visto dos modelos matemáticos que intentan dar una explicación al mecanismo que rige la dinámica de la población de células T durante una infección aguda. Ambos modelos están basados en hechos biológicos conocidos y contrastados y, presentan ecuaciones simples y un número de parámetros reducido, pero suficiente para poder reproducir distintas situaciones sin necesidad de otros elementos. Pero no solo podemos reproducir hechos observados con estos modelos, también podemos predecir, mediante el ajuste de los parámetros correspondientes, cómo se comporta la población ante determinadas tesituras. Como el caso de que la población presente una afinidad baja con el patógeno o que la cantidad de activaciones sea menor a la cantidad de encuentros de la célula T con el \textit{antígeno}. Esto hace que el estudio pueda avanzar sin necesidad de un laboratorio ni experimentos físicos. Otro de los puntos a remarcar de estos modelos es que no son cerrados. Es decir, proporcionan un entorno abierto en el que incluir nuevo conocimiento biológico.



\begin{itemize}
	\item Resultados generales de los dos modelos: ambos reproducen hechos observados y hacen predicciones sobre el comportamiento de las poblaciones.
	
	\item Se pueden reproducir situaciones diversas con los modelos mediante el ajuste adecuado de los parámetros. Lo que permite el estudio de las poblaciones sin necesidad de un laboratorio.
	
	\item En el caso de que se incluya el trabajo de la red neuronal: poner las conclusiones e insistir ahí con el trabajo futuro.
\end{itemize}

