\chapter{Conclusiones y Trabajo Futuro}
\label{cap:conclusiones}

%Conclusiones del trabajo y líneas de trabajo futuro.

%Antes de la entrega de actas de cada convocatoria, en el plazo que se indica en el calendario de los trabajos de fin de máster, el estudiante entregará en el Campus Virtual la versión final de la memoria en PDF. En la portada de la misma deberán figurar, como se ha señalado anteriormente, la convocatoria y la calificación obtenida. Asimismo, el estudiante también entregará todo el material que tenga concedido en préstamo a lo largo del curso.

Tras la lectura de este documento hemos podido comprobar que las matemáticas pueden convertirse en herramientas muy potentes también fuera de su ámbito más teórico, algo que aún las enriquece más. Concretamente, las colaboraciones con biólogos cada vez son más frecuentes y, a pesar de ser ciencias muy distintas, se han obtenido resultados relevantes como el que describíamos a lo largo de estas páginas. Aún son numerosas las preguntas que quedan por resolver en el campo de la biología, es precisamente en esos puntos donde las matemáticas, gracias a su poder de abstracción y simplificación, pueden formular modelos que no solo reproduzcan hechos observados, sino que infieran nuevo conocimiento. Estas nuevas vías de investigación permiten, en muchos casos, avanzar en el estudio de una línea de pensamiento y descartar otras, dando cabida al progreso científico.

Los modelos propuestos en los Capítulos \ref{cap:descripcionTrabajo} y \ref{cap:modeloMacroscopico} se presentan como posibles explicaciones a un mecanismo biológico aún por determinar, como es la dinámica de población de las células T durante una infección aguda. Ambos modelos constituyen modelos robustos, flexibles y bien fundamentados, pues sus hipótesis las establecen evidencias biológicas. Gracias a estos modelos somos capaces de reproducir y predecir el comportamiento de las células T durante una infección aguda en distintas situaciones mediante la variación del valor de sus parámetros, sin necesidad de costosos experimentos ni un laboratorio, y desde dos puntos de vista diferentes, el microscópico y el macroscópico. Pero, sin duda, una de las características más importantes de estos modelos es que no son modelos cerrados, ambos están abiertos a la inclusión de nuevo conocimiento biológico. 

Del modelo microscópico destacamos su contraposición a la hipótesis de que el desarrollo de la vida de una célula T viene determinado por la \textit{estimulación antgénica} recibida durante su activación. De esta manera, las células generadas en la \textit{expansión clonal} tendrían un control muy limitado sobre su elección entre división o apoptosis. Sin embargo, el modelo propuesto expone que los encuentros de una célula T con el \textit{antígeno} son transmitidos a las células hijas por medio de receptores de membrana, que se reparten durante la división celular. Esto permite a las nuevas células integrar este conocimiento con su propia experiencia con el \textit{antígeno}, posibilitando que células que comparten un mismo ancestro tomen decisiones distintas. Con ello, se pone de manifiesto que la heterogeneidad de decisiones observada durante una \textit{respuesta inmune} puede ser explicada mediante un algoritmo determinista e independientemente ejecutado por las células T. La respuesta que dan las células T es específica para un \textit{antígeno}, pero no para un patógeno\footnote{Un \textit{antígeno} específico puede estar presente en diversos \textit{patógenos}, que pueden ser muy heterogéneos en términos de ratio de crecimiento o mecanismos para escapar a la \textit{respuesta inmune}.}. Esto explica el hecho de que los mecanismos de reconocimiento de \textit{patógenos} no se detallen en el modelo, dotando al mismo de capacidad de adaptación a distintos ataques \citep{JTB}.

Por su parte, el modelo macroscópico observa la dinámica de población de las células T (\textit{expansión y contracción clonal}) de manera colectiva. Modelando su comportamiento mediante ecuaciones diferenciales de segundo orden, sustentadas por dos propiedades poblacionales: la elasticidad y la inercia. Este modelo arroja resultados interesantes. En concreto, propone que el mecanismo de identificación de la población objetivo de las células T viene determinada por la tasa de crecimiento de esta. Es decir, aquellas poblaciones que crecen muy rápidamente son consideradas como \textit{patógenos}, mientras que aquellas que limitan su ratio de crecimiento son toleradas. Esto explica el hecho paradójico de que un crecimiento lento sea la estrategia de evasión de algunas células tumorales o virus como el de la hepatitis C \citep{arias2015growth}. 

Otra de las cuestiones que aún busca respuesta es la planteada en el Capítulo \ref{cap:redNeuronal}. En este se buscaba establecer una correspondencia entre los parámetros de los dos modelos vistos en los capítulos anteriores, puesto que ambos muestran comportamientos poblacionales compatibles. El modelo microscópico se caracteriza por representar características biológicas explícitas de las células mediante parámetros estructurales, cuyo valor permanece fijo durante la simulación. Sin embargo, el significado de los parámetros del modelo macroscópico, referentes a las características de inercia y elasticidad de la población de células T, carecen de un significado biológico claro. A pesar de ello, una de las ventajas de este modelo es el reducido número de parámetros que tiene. Es por ello que encontrar una correspondencia de parámetros entre ambos modelos sería de gran utilidad para encontrar los parámetros del modelo microscópico que se corresponden con una cierta \textit{respuesta inmune}\footnote{De esta manera, la respuesta puede simularse utilizando el modelo macroscópico, cuyos parámetros son más fáciles de ajustar y, posteriormente, establecer los correspondientes parámetros en el modelo microscópico, los cuales son fácilmente interpretables.}. A lo largo del Capítulo  \ref{cap:redNeuronal} se detalla una primera aproximación para la resolución de esta compleja cuestión. Los resultados obtenidos por la red neuronal no son despreciables, pero aún insuficientes para poder deducir una correspondencia formal entre ambos modelos. Es por ello que queda a la espera de nuevo estudio e investigación.

