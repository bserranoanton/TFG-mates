
\chapter{Simulaciones}
\label{cap:simulaciones}

A lo largo de este capítulo veremos con detalle las simulaciones que se han realizado del modelo visto en la Sección \ref{sec:modelo}, con algunas simplificaciones. Reproduciremos el comportamiento individual de cada célula y veremos, una vez completada la simulación, el efecto global de estas decisiones individuales. Como ya se ha comentado en el capítulo anterior, el modelo permite ajustar los parámetros de tal manera que se pueda ajustar a distintas situaciones. Concretamente, a continuación, se presentarán situaciones de tolerancia al \textit{patógeno}, correspondientes a \textit{patógenos} cuya población crece más despacio y consigue engañar a nuestras células T, situaciones de intolerancia, aquellas en las que conseguimos ganarle la batalla al \textit{patógeno}, y el caso de poblaciones de células T con distinta afinidad al \textit{patógeno}.

Con ayuda de un pseudocódigo veremos algunos detalles sobre de la implementación de las simulaciones, el código, realizado en Matlab puede verse en el Apéndice \ref{Appendix:A}.


\section{Modelo simplificado}

Para las simulaciones hemos optado por un sistema simplificado al propuesto en la Sección \ref{sec:modelo}, de tal manera que el número de parámetros sea suficiente para no perder la esencia del modelo pero no muy elevado para no distraer al lector con notación engorrosa. Siguiendo con la notación de \ref{sec:modelo}, asumiremos $k=2$. Es decir, hay dos tipos de receptores: \textit{p} (de proliferación) y \textit{d} (de muerte) que controlan
la evolución de los inhibidores de ciclo (Rb) y apóptosis (Bcl-2), respectivamente.

Para las células T efectoras que no forman parte de las células T de memoria (veremos ecuaciones específicas para esta población) asumimos que los receptores de proliferación (\textit{p}) se expresan a partir de las señales que reciben gracias a su TCR y que, simultáneamente, autorregulan su expresión induciendo la producción de receptores tipo muerte (\textit{d}). Luego, las ecuaciones \ref{sist_inhib} y \ref{sist_recep} pueden escribirse como:

\begin{equation}
	\label{sist9_simplif}
	\left\{ \begin{array}{l}
	\dot{c}(t) = -\mu_{pc}p(t) \\
	\dot{a}(t) = -\mu_{da}d(t)  \\
	\dot{p}(t) = \lambda_{Tp}r_{T}(t) - \lambda_{pp}p(t) \\
	\dot{d}(t) = \lambda_{pd}p(t) \\
	\\
	c(0)=c_0 \\
	a(0)=a_0 \\
	p(0)=p_0 \\
	d(0)=d_0 
	\end{array}
	\right.
\end{equation}

Así mismo, hemos simulado de manera conjunta este Sistema \ref{sist9_simplif} y la Ecuación \ref{sist_pat_T} para el caso de las células T de memoria. La dinámica de estas células viene dada por el mismo Sistema \ref{sist9_simplif}, en el que se ha tenido en cuenta que $d=0$, puesto que nos centramos solamente en el inhibidor del ciclo celular (recordemos que las células T de memoria no mueren durante la \textit{contracción clonal}). Así las cosas, las ecuaciones que rigen el algoritmo de decisión para células T de memoria viene dado por: 

\begin{equation}
	\label{sist15_simplif}
	\left\{ \begin{array}{l}
	\dot{c}(t) = -\mu_{pc}p(t) \\
	\dot{p}(t) = \lambda_{Tp}r_{T}(t) - \lambda_{pp}p(t) \\
	\\
	c(0)=c_0 \\
	p(0)=p_0 \\
	\end{array}
	\right.
\end{equation}

Ahora que ya tenemos las ecuaciones del modelo estamos casi listos para la simulación, falta establecer el valor concreto que daremos a los parámetros. Esta no es una tarea sencilla, pues muchos de ellos son específicos para un \textit{patógeno} concreto, lo que hace que obtengamos un resultado u otro dependiendo de estos. Además, hay que ajustarlos \textit{a ojo}, pues no está escrito en ningún gen su valor concreto. En la Tabla \ref{tabla:param} tenemos los parámetros elegidos para la primera simulación: tolerancia al patógeno. En lo que sigue veremos cómo cambia la elección de parámetros. A continuación se presentan los detalles más básicos de la implementación.



\section{Pseudocódigo}

Con ánimo de aclarar los aspectos más básicos de la implementación, se explican los pasos seguidos para el desarrollo de la misma. A modo de complemento, el Algoritmo \ref{algo:pseudocodigo} contiene un pseudocódigo muy sencillo con los detalles claves y prácticamente independientes del lenguaje de programación que se haya utilizado. El código completo, realizado en Matlab, puede verse en el Apéndice \ref{Appendix:A}.

Veamos, paso por paso, cómo se simula el modelo: 

\begin{enumerate}
	\item Comenzamos la simulación en un tiempo inicial $t=0$ y acabamos en un tiempo final $T_{final}$ que se establecerá una vez las células T efectoras han desaparecido. 
	
	\item Para cada tiempo $t$, se calcula la cantidad de \textit{patógeno} disponible, $Y$. 
	
	\item En función de $Y$, y para cada célula T de la población, se calcula la cantidad de \textit{patógeno} que está a su alcance y se resuelve el sistema de ecuaciones correspondiente para conocer la cantidad de Rb ($c$) y Bcl-2 ($a$) activa en ese instante. En función de esto se desencadenará la división celular, si $c = 0$, o el suicidio de la célula, si $a = 0$.
	
	\item Si la célula va a dividirse se generan dos células hijas con los parámetros correspondientes al TCR, recordemos que la cantidad de receptores de la célula madre se divide entre las dos hijas de manera asimétrica, y los parámetros iniciales, para que pueda comenzar su fase de \textit{decisión}. Se sigue en el paso 6.
	
	\item Si por el contrario la célula comete suicidio, se eliminará de la población. 

	\item Se contempla la siguiente célula de la población y se vuelve a 3.
	
	\item Se actualiza el tiempo para la siguiente iteración y se vuelve a 1.
\end{enumerate}


\begin{algorithm}
	\caption{Algoritmo de la decisión. Células T.}
	\label{algo:pseudocodigo}
	\begin{algorithmic}[1]
		
		% ENTRADA / SALIDA
		
		\State Inicialización de parámetros según \ref{tabla:param}
		\State $t = 0;$ \Comment{t será el tiempo por el que vamos simulando}
		
		\While{ $t\ < \ T_{final}$}
		  
		\State $Y = Y_{init}*e^{t*(\alpha - N*\beta)};$ \Comment{Calculamos Y con la solución explícita de \ref{sist_pat_T}}
		
		\For{$nCell; nCell++; N$} \Comment{Para cada célula T de la población}
			\State $ r_{T}=\rho*Y;$ \Comment{Ecuación \ref{ec:rhotau}}
			\If{$efectora(nCell)$} \Comment{Si es una célula T efectora}
				\State Se resuelve \ref{sist9_simplif}
				\If{$a \leq 0 $}
					\State La célula $nCell$ se elimina de la población
				\ElsIf{$c \leq 0$}
					\State La célula $nCell$ se divide
					\State Las condiciones iniciales de las células hijas vienen determinadas por $a_0, c_0$ y \ref{sist:div_sim}
				\EndIf
			
			\ElsIf{$memoria(nCell)$} \Comment{Si es una célula T de memoria}
				\State Se resuelve \ref{sist15_simplif}
				\If{$c \leq 0$}
					\State La célula $nCell$ se divide siguiendo el mismo procedimiento que la división de una célula T efectora. 
				\EndIf
			\EndIf
		\EndFor
		
		\State Se actualiza el número de células de la población.
		\State $t = t + t_{next};$
		
		\EndWhile
		
	\end{algorithmic}
\end{algorithm}

En este pseudocódigo se ha detallado cuáles son las ecuaciones involucradas en cada paso. A continuación exponemos algunas particularidades de la simulación: hemos omitido que cuando las condiciones son $a > 0$ y $c > 0$, en el caso de las células T efectoras y $c > 0$, en el caso de las células T de memoria, la célula permanece en la fase de decisión pero actualiza sus condiciones para la siguiente iteración según los resultados que ha obtenido en la iteración actual. También hay que tener en cuenta que la división celular y el proceso de apóptosis no se llevan a cabo de manera inmediata, tienen un tiempo $t_{cycle}$ y $t_{apo}$, respectivamente, por lo que el número total de células en la población debe actualizarse cuando toque y no antes de que ninguna de estas dos fases haya finalizado. Otro aspecto que hemos supuesto es que el parámetro $\gamma$ que aparecía en la Ecuación \ref{ec:rhotau} es $\gamma = 1$. Es decir, suponemos que todo encuentro del TCR de la célula T con el antígeno va a desencadenar una activación. El parámetro $\rho$ debe ser calculado de tal manera que todas las células T tengan las mismas posibilidades a la hora de \textit{obtener su parte de patógeno}, en la implementación real se usó un vector de números aleatorios entre 0 y 1 normalizado por el número total de células T.

Buena parte de la notación usada en el Algoritmo \ref{algo:pseudocodigo} ya ha sido introducida a lo largo de este trabajo, pero volvemos a insistir en que $Y$ representa el número de moléculas del patógeno, mientras que $N$ la cantidad total de células T, incluyendo las efectoras y las de memoria. Sin embargo, en la implementación real, en la línea 4 del pseudocódigo, el $N$ utilizado es solamente el número total de células T efectoras, sin contar las de memoria \footnote{Esto se ha hecho así porque el proceso que siguen las célualas T de memoria es más complejo que lo que se recoge en el modelo. Estas células al cabo de un tiempo se desactivan y para que tengan un efecto sobre el patógeno deben volver a activarse. Para intentar hacer el modelo lo más sencillo posible se ha optado por hacer que las únicas células que combaten al patógeno sean las T efectoras.}.

\section{Resultados y análisis}

En esta sección  veremos los resultados de algunas simulaciones y el por qué de estos resultados. Empezaremos por las dos situaciones básicas que se pueden dar en una infección: que logremos vencer al atacante o que, por el contrario, seamos vencidos, y acabaremos mostrando el resultado de diversas simulaciones cuando la afinidad por el \textit{patógeno} de las células T va variando.

\subsection{Intolerancia al patógeno}
\label{sim:intoler}

La primera de nuestras simulaciones puede verse en la Figura \ref{fig:intolerance}, esta muestra el caso correspondiente a la elección de parámetros que se recoge en la Tabla \ref{tabla:param}. Estamos ante un caso de intolerancia al patógeno, puesto que las células T son capaces de eliminarlo por completo. Veámoslo con más detalle: el patógeno, representado con un línea roja, crece rápidamente, debido a la elección de una tasa de crecimiento, $\alpha$, elevada. Una vez que las células T son conscientes de la rápida proliferación de un agente no deseado, su número comienza a crecer. Sin embargo, como ya habíamos comentado anteriormente, esto se produce con cierto retraso tras la aparición del patógeno. Lo que estamos describiendo es la conocida \textit{expansión clonal}. Este crecimiento de células T provoca que el término que acompaña a $\beta$ en la Ecuación \ref{sist_pat_T} comience a ser más grande que el acompañado por $\alpha$ en esta misma ecuación, provocando así que la derivada de $y$ se haga negativa y, por tanto, el número de moléculas del patógeno comience a decrecer. Debemos mencionar que el número de células T necesarias para eliminar el patógeno viene regulado por el parámetro $\beta$, si este fuera más grande, es decir, las células T fueran más dañinas con el patógeno, veríamos una curva azul con un máximo mas pequeño que el de la Figura \ref{fig:intolerance}. (EN REALIDAD EN ESTA FIGURA LA GRÁFICA ESTÁ NORMALIZADA, SERÍA MEJOR PONER LA OTRA SIN NORMALIZAR?)Pero ¿qué pasaría si la fuerza de estas células T no fuera suficiente? El patógeno crecería de manera exponencial, a mucha más velocidad que las células T, de forma que estas no podrían llegar a acabar con él (un \textit{spoiler} de lo que viene en la sección siguiente).


\begin{table}[h]
	\begin{center}
		\begin{tabular}{>{\centering\arraybackslash}m{2cm} >{\arraybackslash}m{3cm} >{\arraybackslash}m{7cm} }%{|l|l|l|}
			\hline
			\multirow{9}{*}{} & $t_{cycle} = 0.15$               & Duración de la fase de ciclo.                             \\ \cline{2-3}
			& $t_{apo} = 0,2$                  & Duración de la fase de apóptosis.                         \\ \cline{2-3}
			& $t_{next} = 0,3$                 & Duración del paso en la simulación.                       \\ \cline{2-3}
			& $a_0 = 0,3$                      & Cantidad inicial de Bcl-2 para células T efectoras.       \\ \cline{2-3}
			Variables         & $c_0 = 0,08$                     & Cantidad inicial de Rb para células T efectoras.          \\ \cline{2-3}
			& $c_0^{mem} = 0,04$               & Cantidad inicial de Rb para células T de memoria.         \\ \cline{2-3}
			& $N_{ini} = 25$                   & Número inicial de células T naïve.                        \\ \cline{2-3}
			& $Y_{ini} = 5$                    & Número inicial de moléculas del patógeno.                 \\ \cline{2-3}
			& $r_p, r_d = 0$                   & Número inicial de receptores de membrana $p$ y $d$.       \\ \hline
			\multirow{2}{*}{Patógeno}  & $\alpha = 6$                    & Tasa de proliferación.                                    \\ \cline{2-3}
			& $\beta = 0,04$                    & Tasa de muerte por linfocito.                             \\ \hline
			\multirow{5}{*}{} & $\lambda_{pd} = 0,05$            & Tasa de cambio del receptor $R_d$ por cada señal $R_p$.   \\ \cline{2-3}
			& $\lambda_{Tp} = 6*10^{-5}$       & Tasa de cambio del receptor $R_p$ por cada señal del TCR. \\ \cline{2-3}
			Células T  efectoras       & $\lambda_{pp} = 0,5*10^{-4}$     & Tasa de cambio del receptor $R_p$ por cada señal $R_p$.   \\ \cline{2-3}
			& $\mu_{pc} = 15$                 & Tasa de cambio de Rb por cada señal del TCR.              \\ \cline{2-3}
			& $\mu_{da} = 10$                 & Tasa de cambio de Bcl-2 por cada señal del TCR.           \\ \hline
			\multirow{4}{*}{} & $\lambda_{Tp}^{mem} = 10^{-5}$   & Igual que $\lambda_{Tp}$, para células T de memoria.      \\ \cline{2-3}
			Células T de memoria        & $\lambda_{pp}^{mem} = 2*10^{-2}$ & Igual que $\lambda_{pp}$, para células T de memoria.      \\ \cline{2-3}
			& $\mu_{pc}^{mem} = 13$           & Igual que $\mu_{pc}$, para células T de memoria.          \\ \cline{2-3}\hline
		\end{tabular}
		\caption{Tabla de variables y parámetros.}
		\label{tabla:param}
	\end{center}
\end{table}


Prestemos atención ahora al comportamiento de las células T de memoria: por la sección anterior, ya sabíamos que las células T efectoras y las de memoria iban a constituir poblaciones distintas, puesto que las ecuaciones que rigen sus dinámicas son distintas. La principal diferencia es que las células T de memoria no se suicidan una vez el patógeno ha desaparecido, sabemos que permanecen con la información necesaria para atacar al patógeno más rápidamente en caso de reaparición. Vemos cómo estas células de memoria aumentan su población tras la aparición del patógeno, no vemos un crecimiento tan grande. Su población queda reducida a un $5-10\%$ de la población de células T.


\begin{figure}[t]
	\centering
	\includegraphics[width=0.7\textwidth]{Imagenes/Simulaciones/intolerance}
	\caption{Simulación: caso de intolerancia al patógeno.}
	\label{fig:intolerance}
\end{figure}

\subsection{Tolerancia al patógeno}
\label{sim:toler}

En el caso anterior hemos visto una simulación de intolerancia al patógeno. Esto es, las células inmunes consiguen derrotarlo. Sin embargo, esto no es siempre así. Existen virus como PONER EJEMPLO SI SE SABE que han desarrollado una estrategia para intentar sobrevivir lo máximo posible dentro de nuestro cuerpo, lo que hacen es crecer a un ritmo muy lento, de esta manera \textit{sigilosa} engañan a las células T, haciéndolas creer que ha sido eliminado y provocando que estas células inmunes se suiciden. La Figura \ref{fig:tolerance} ilustra esta situación.

Como vemos, las células T comienzan la \textit{expansión clonal}. Este aumento de población inmune hace que la población del patógeno se vea afectada rápidamente, recordemos que su factor de crecimiento, $\alpha$, es pequeño ahora. De esta manera, las células inmunes perciben que el patógeno ha sido eliminado con éxito, puesto que el número de células del mismo es muy pequeño, y comienzan la \textit{contracción clonal}, haciendo que su población baje hasta desaparecer. Sin embargo, el patógeno no ha sido erradicado por completo, aún quedaban algunos organismos, imperceptibles para las células inmunes, Ahora que no hay atacantes, el \textit{patógeno} puede reproducirse sin problema. Es por esto que su población crece de manera exponencial. En poco tiempo estos \textit{patógenos} \textit{astutos} pueden tomar el control del otro organismo. 

En cuanto a las células T de memoria, vemos como crecen con la presencia del patógeno y se estabilizan cuando la población de células T efectoras llega a cero. Esto es así puesto que las células T de memoria no continúan reproduciéndose en ausencia de células T efectoras, a pesar de la presencia de patógeno. PREGUNTAR RAZÓN 

\begin{figure}[t]
	\centering
	\includegraphics[width=0.7\textwidth]{Imagenes/Simulaciones/tolerance}
	\caption{Simulación: caso de tolerancia al patógeno.}
	\label{fig:tolerance}
\end{figure}

\subsection{Simulaciones con distintas poblaciones de células T}

En esta sección veremos cómo se comportan distintas poblaciones de células T efectoras frente a un mismo patógeno. Estas poblaciones están diseñadas para que tengan afinidades distintas con el \textit{patógeno}. Un caso interesante es ver qué ocurre cuando alguna de estas poblaciones desaparece. 

Comencemos mirando la Figura \ref{fig:tresClones}, para esta simulación hemos tomado tres poblaciones con distinta afinidad, $\lambda_{Tp}$, al patógeno. Tenemos el clon 0 con la afinidad más alta y el clon 2 con la más baja. La diferencia en cuanto a expansión es considerable, la población más afín al patógeno es la que se reproduce a mayor velocidad. Esto parece lógico, puesto que es la más preparada para combatir al patógeno, se denomina \textit{población inmunodominante}. Además, el Sistema 	\ref{sist9_simplif} así lo dictamina: la ecuación $\dot{p}(t) = \lambda_{Tp}r_{T}(t) - \lambda_{pp}p(t)$ propicia un mayor crecimiento cuanto más alto es el valor $\lambda_{Tp}$, puesto que provoca que la derivada de $c$ se haga más negativa y se llegue antes al límite $c = 0$ que desencadena la división celular.

\begin{figure}[t]
	\centering
	\includegraphics[width=0.7\textwidth]{Imagenes/Simulaciones/tresClones}
	\caption{Simulación:Distintas poblaciones de células T con distintas afinidades al patógeno.}
	\label{fig:tresClones}
\end{figure}


Pero... ¿qué pasaría si esta \textit{población inmunodominante} desapareciera? Una posible explicación nos la da la Figura \ref{fig:dosClones}. En ella, podemos ver que el modelo sugiere que las \textit{poblaciones subdominantes} se expanden en mayor medida que antes para suplir la ausencia de la \textit{inmunodominante} y controlar la infección. No debemos olvidar que la afinidad que tienen estas poblaciones al patógeno es menor y esto hace que el patógeno pueda crecer más en el mismo periodo de tiempo, esto significa que la infección será más aguda y durará más tiempo. 


\begin{figure}[t]
	\centering
	\includegraphics[width=0.7\textwidth]{Imagenes/Simulaciones/dosClones}
	\caption{Simulación:Distintas poblaciones de células T con distintas afinidades al patógeno. Clones subdominantes.}
	\label{fig:dosClones}
\end{figure}

Para finalizar veamos el comportamiento del clon 2 cuando el resto de clones han desaparecido. Como es de esperar, ocurre algo similar a lo que veíamos en la Figura \ref{fig:dosClones}. En este caso el clon 2 debe hacer un esfuerzo mayor (reproducirse más) para mantener la infección controlada. Comportamiento ilustrado en la Figura \ref{fig:unClon}.

Estas simulaciones que hemos visto ponen de manifiesto la importancia de las células T de memoria. En una situación donde las células T efectoras no presentan una afinidad al \textit{patógeno} muy elevada las consecuencias pueden ser muy graves, pues la infección se alarga y las células T no son especialmente dañinas con el agente externo. Sin embargo, si contamos con células T de memoria que guardan información relevante para combatir a ese agente, nuestro organismo dispondrá de una situación mucho más privilegiada, ya que se podrá actuar más rápidamente con células que disponen de alta afinidad con el \textit{patógeno} y desencadenarán, por tanto, un ataque mucho más nocivo.

\begin{figure}[t]
	\centering
	\includegraphics[width=0.7\textwidth]{Imagenes/Simulaciones/unClon}
	\caption{Simulación:Distintas poblaciones de células T con distintas afinidades al patógeno. Clon subdominante.}
	\label{fig:unClon}
\end{figure}

