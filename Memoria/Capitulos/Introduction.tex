\chapter{Introduction}
\label{cap:introduction}



The year 2018 was proclaimed International Year of Mathematical Biology by two scientific societies: the European Mathematical Society (EMS) and the European Society for Mathematical and Theoretical Biology (ESMTB). This celebration intended to highlight the importance of the applications of mathematics to biology and life sciences and to encourage their interaction\footnote{\url{https://www.icmat.es/divulgacion/Material_Divulgacion/miradas_matematicas/06.pdf}}. Today, life sciences are increasingly using mathematics, ranging from the use of dynamic systems and statistics to population and disease spread models. In this context, models play an important role, as they are simplified representations of the structure and functioning of a given biological system or process, using mathematical language to express the relationships between variables\footnote{\url{http://www.blogsanidadanimal.com/2018-el-ano-internacional-de-la-biologia-matematica/}}. The appropriate use of models allows us to advance beyond what mere intuition suggests and provides us with useful information that would otherwise be difficult to collect, either because of the high economic cost of the experiments, the time it takes to carry them out or the amount of data to be examined, among other reasons. But it is not only experts who benefit from the simplifying power of mathematical models. During the current COVID-19 health crisis, mathematical models have been used to predict the spread of the virus and to inform society of the risk of this pandemic\footnote{Mathematical models on COVID-19 growth curve in \textit{The Washington Post}: \url{https://www.washingtonpost.com/graphics/2020/world/corona-simulator/}}. 



Throughout this document, we will focus on the field of immunology. It is interesting to note that the cells that make up the immune system are not regulated by a coordinating organ \citep{arias2016emergent}. Immune cells move freely through the body and lead an independent life. However, they are capable of displaying collective behaviour, as is the case with the response to infectious agents. In this defensive function, T cells play a very important role. When an infection is detected, the population of this type of cells grows by several orders of magnitude in a few days and, once the infectious agent has disappeared, population levels are restored through the suicide (apoptosis) of a large part of the population generated. The decision mechanism between division or apoptosis taken by T cells during the immune response still has many questions. In the following chapters, we will present two mathematical models, based on differential equations, which attempt to shed light on this phenomenon. The first one, which can be seen in the Chapter \ref{cap:descripcionTrabajo}, deals with this issue from a microscopic point of view. That is, it proposes a decision algorithm implemented by each cell. Meanwhile, the equations of the second model, set out in the Chapter \ref{cap:modeloMacroscopico}, describe the dynamic behaviour of the entire population of T cells, based on two main characteristics attributed to that population: elasticity and inertia. In both cases, numerical simulations of these models are performed. These simulations represent different situations that can occur during infection. Among them, it is interesting to distinguish between the situation of intolerance to the pathogen, in which case the immune cells manage to control the infection and eliminate the infectious agent, or the situation of tolerance to the pathogen, in which it is the latter who ends up taking control of the organism. It also discusses what happens when we have populations of T cells with different affinities to the pathogen. The relationship between the properties of both models is an interesting topic, which is addressed in the last chapter (Chapter \ref{cap:redNeuronal}). As we will see, both give rise to results that are not only compatible but complementary. 



\section{Objectives}


This Final Project focuses on the study of the population dynamics of T cells in the face of acute infection and, more specifically, on the study of two mathematical models that aim to respond to the mechanisms that govern this behaviour.

To approach this project, the following objectives were determined:

\begin{itemize}
	
	\item Study of the basics of the immune system, focusing on the role that T cells play during an \textit{immune response}.
	
	\item Study and understanding of the mathematical models detailed in this document and their importance in the field of biology.
	
	\item Implementation of the code needed to simulate different T cell behaviours based on these models and analysis of these behaviours. 
	
	\item Obtaining a first approximation to establish a correspondence between the parameters of the two models studied. 
\end{itemize}

\section{Work plan}

To carry out this work, various milestones were established throughout the academic year. The first months were devoted to a prior review of the underlying biological concepts. These included basic notions about the immune system and a more detailed study of T cell behaviour. This constitutes a fundamental part of the work, as the models cannot be fully understood if they are not looked at from the biological problem to which they are trying to respond. Once the biological basis was established, the study of the models could begin. The first one to be studied would be the microscopic model shown in \cite{JTB}. Different simulations of the model were established, which would complement the theory seen. The second model, the macroscopic model \citep{arias2015growth}, would be studied later, with the purpose of being able to relate it to the previous one. 

Once both models had been reviewed and the corresponding simulations had been carried out, the possibility was opened up of trying to establish a correspondence between the parameters of both models by implementing a neural network. The latter would put an end to the content of this Bachelor's theses.


\section{Document structure}

This work is divided into four well-differentiated parts, but with the same purpose, the study of T cells and their population dynamics during an acute infection. 

\begin{enumerate}
	\item The context of the document is covered in Chapter \ref{cap:estadoDeLaCuestion}. In particular, Section \ref{sec:cuestInmuno} covers basic notions of immunology, which allows the reader to proceed through the following chapters without any terminological impediment, as far as biological issues are concerned. This section is intended to give a very basic overview of the immune system. It starts with the simplest mechanisms, referring to the \textit{innate immune system} (Section \ref{sub:sistInmInnato}), to the most complex ones, referring to the \textit{adaptive immune system} (Section \ref{sub:sistInmAdap}). The aspects of the immune response involving T cells, such as their activation and performance or immune memory, are discussed in more detail (Section \ref{Tcell}).
	
	Furthermore, Section \ref{sec:coop} deals with the role of mathematical models in the field of biology. Specifically, in Section \ref{cuestionAmodelizar}, we focus on the case of our study, the different mathematical models formulated for the dynamics of T cells during an acute infection. 
	
	\item The theoretical framework of the proposed microscopic model for the problem of deciding between T cell division and apoptosis is presented in Chapter \ref{cap:descripcionTrabajo}. In Section \ref{sec:hip_bio} the biological hypotheses on which the model is based are detailed. Those hypotheses constitute contrasted and observed facts in the field of biology. The model itself can be seen in Section \ref{sec:modelo}, which details the notation that will be followed by the rest of the document and the first-order differential equations that give rise to the algorithm. The last section of this chapter, Section \ref{sec:modeloPatCelT}, introduces a differential equation for the population dynamics of the pathogen and its relationship to the number of available T cells. The equation establishes the interaction between the two populations.
		
	In Chapter \ref{cap:simulaciones} the simulations corresponding to a simplified case of the previous model are presented (Section \ref{sec:modelo_simplif}) and the basic details of its implementation are explained (Section \ref{sec:implem_pseudo}). The results of the simulations are described in Section \ref{sec:simulacionesMicro}. These simulations correspond to cases of intolerance and tolerance to the pathogen (Sections \ref{sim:intoler} and \ref{sim:toler}, respectively), as well as the case of the immune response with populations of T cells with different affinities to the pathogen (Section \ref{sim:difPoblacionesT}).
	
	\item The macroscopic model is studied in Chapter \ref{cap:modeloMacroscopico}. The second-order differential equations of this model govern the population dynamics of T cells and the pathogen collectively, unlike the microscopic model, whose algorithm was defined for each cell. This model is based on two characteristics of the T cell population behaviour during an immune response: elasticity and inertia (Section \ref{sec:iner_elast}). 
	
	In addition to proposing a theoretical model, the numerical simulations corresponding to the model are also carried out in Section \ref{sec:simu_macro}. These include the cases of intolerance and tolerance to the pathogen (Sections \ref{sub:simMacroIntoler} and \ref{sub:simMacroToler}, respectively) and, for the macroscopic non-dimensional model, the relevance of the value of its two parameters in the regions of tolerance and intolerance is studied (Section \ref{sub:reg_tolerIntolerMacro}).

	\item Then, after studying the models proposed in Chapters \ref{cap:descripcionTrabajo} and \ref{cap:modeloMacroscopico}, and after comparing their results, a parameter correspondence is sought between both models in Chapter \ref{cap:redNeuronal}. For this purpose, a neural network which is capable of performing ``the inverse function'' to the code referring to the simulations of the macroscopic model is implemented. That is to say, given the results of a simulation, the network can predict the value of the parameters necessary to obtain that same result. The construction of the dataset and the implementation of the network can be seen in the Section \ref{sec:conjDatos_entreRed}. The results obtained and an example of execution, using as input the results of a simulation of the microscopic model, can be seen in the Section \ref{sec:resultadoRed}.
	

\end{enumerate}

Finally, Chapter \ref{cap:conclusions} offers a brief conclusion of the work carried out. To complement the project, the main code of the model simulations (both those of Chapter \ref{cap:simulaciones} and \ref{cap:modeloMacroscopico}) has been included in Appendix \ref{Appendix:A}.

