\chapter{Introducción}
\label{cap:introduccion}

\chapterquote{Frase célebre dicha por alguien inteligente}{Autor}



\section{Motivación}
La habilidad de nuestro sistema inmune (SI) para protegernos de los patógenos es ciertamente apasionante. Las células inmunes deben saber cómo diferenciar a las células amigas de las enemigas y cómo y dónde actuar.
Son diversas las amenazas a las que el SI tiene que enfrentarse: pueden ser de naturaleza biológica (agentes patógenos), físico-químicas (como contaminantes o radioaciones) o internas (por ejemplo, las células cancerosas).

A lo largo de los años, muchas preguntas sobre el funcionamiento del SI han sido respondidas, pero aún quedan muchas otras por responder: ¿Quién regula la actuación del SI? ¿Qué influye en la respuesta inmune?, entre otras.  

Si bien parece natural pensar que hay un órgano que actúa de director, ese
órgano, si existe, aún no se ha encontrado. Incita, por tanto, a considerar, que las
células inmunes basan su actuación en la información local que encuentran a su
alrededor. Y sobre esta suposición construiremos un modelo que describa las
dos actuaciones básicas, división y muerte celular, que desarrollan las células de
nuestro estudio: las células T.

De este comportamiento, destacamos la simplicidad: las células T tienen un número muy limitado de opciones y estas vienen determinadas por el ambiente en el que están.

A pesar de la descentralización en sus tareas y de la aparente sencillez y poco margen de maniobra, es asombroso que el resultado que nos ofrece el SI sea tan eficiente.

\section{Objetivos}
Descripción de los objetivos del trabajo.


\section{Plan de trabajo}
Aquí se describe el plan de trabajo a seguir para la consecución de los objetivos descritos en el apartado anterior.



\section{Explicaciones adicionales sobre el uso de esta plantilla}
Si quieres cambiar el \textbf{estilo del título} de los capítulos, edita \verb|TeXiS\TeXiS_pream.tex| y comenta la línea \verb|\usepackage[Lenny]{fncychap}| para dejar el estilo básico de \LaTeX.

Si no te gusta que no haya \textbf{espacios entre párrafos} y quieres dejar un pequeño espacio en blanco, no metas saltos de línea (\verb|\\|) al final de los párrafos. En su lugar, busca el comando  \verb|\setlength{\parskip}{0.2ex}| en \verb|TeXiS\TeXiS_pream.tex| y aumenta el valor de $0.2ex$ a, por ejemplo, $1ex$.

TFMTeXiS se ha elaborado a partir de la plantilla de TeXiS\footnote{\url{http://gaia.fdi.ucm.es/research/texis/}}, creada por Marco Antonio y Pedro Pablo Gómez Martín para escribir su tesis doctoral. Para explicaciones más extensas y detalladas sobre cómo usar esta plantilla, recomendamos la lectura del documento \texttt{TeXiS-Manual-1.0.pdf} que acompaña a esta plantilla.

El siguiente texto se genera con el comando \verb|\lipsum[2-20]| que viene a continuación en el fichero .tex. El único propósito es mostrar el aspecto de las páginas usando esta plantilla. Quita este comando y, si quieres, comenta o elimina el paquete \textit{lipsum} al final de \verb|TeXiS\TeXiS_pream.tex|

\subsection{Texto de prueba}


\lipsum[2-20]