\chapter{Introducción}
\label{cap:introduccion}

%\chapterquote{Frase célebre dicha por alguien inteligente}{Autor}

El año 2018 fue proclamado Año Internacional de la Biología Matemática por dos sociedades científicas: la \textit{European Mathematical Society} (EMS) y la \textit{European Society for Mathematical and Theoretical Biology} (ESMTB). Con esta celebración se pretendía señalar la importancia de las aplicaciones de las matemáticas en la biología y en las ciencias de la vida y fomentar su interacción\footnote{\url{https://www.icmat.es/divulgacion/Material_Divulgacion/miradas_matematicas/06.pdf}}. En la actualidad, las ciencias de la vida utilizan cada vez más aportaciones matemáticas, que van desde el uso de los sistemas dinámicos y la estadística, a los modelos de población y de propagación de enfermedades. En este contexto, los modelos cobran un papel relevante, puesto que son representaciones simplificadas de la estructura y del funcionamiento de un determinado sistema o proceso biológico, utilizando el lenguaje matemático para expresar las relaciones entre variables\footnote{\url{http://www.blogsanidadanimal.com/2018-el-ano-internacional-de-la-biologia-matematica/}}. El uso adecuado de modelos permite avanzar más allá de lo que la mera intuición sugiere y nos suministra información útil que de otra manera sería difícil recabar, ya sea por el elevado coste económico de los experimentos, por el tiempo que lleva realizarlos o por la cantidad de datos a examinar, entre otras razones. Pero no solo los expertos se benefician del poder de simplificación de los modelos matemáticos. Durante la actual crisis sanitaria de la COVID-19, se han utilizado modelos matemáticos para predecir la propagación del virus y para informar a la sociedad del riesgo de esta pandemia\footnote{Modelos matemáticos sobre la curva de crecimiento del COVID-19 en \textit{The Washington Post}: \url{https://www.washingtonpost.com/graphics/2020/world/corona-simulator/}}. 


A lo largo de este documento nos centraremos en el campo de la inmunología. Es interesante observar que las células que componen el sistema inmune no están reguladas por un órgano coordinador \citep{arias2016emergent}. Estas células se mueven libremente por el organismo y llevan una vida independiente. Sin embargo, son capaces de desplegar comportamientos colectivos, como es el caso de la respuesta ante agentes infecciosos. En esta función defensiva, las células T juegan un papel muy importante. Cuando se detecta una infección, la población de este tipo de células crece varios órdenes de magnitud en pocos días y, una vez desaparecido el agente infeccioso, los niveles de población vuelven a restaurarse mediante el suicidio (apoptosis) de gran parte de la población generada. El mecanismo de decisión entre división o apoptosis que toman las células T durante la respuesta inmune guarda aún muchos interrogantes. En los capítulos que siguen expondremos dos modelos matemáticos, basados en ecuaciones diferenciales, que intentan arrojar luz sobre este fenómeno. El primero de ellos, que puede verse en el Capítulo \ref{cap:descripcionTrabajo}, aborda este asunto desde un punto de vista microscópico. Es decir, se propone un algoritmo de decisión implementado por cada célula. Por su parte, las ecuaciones del segundo modelo, expuesto en el Capítulo \ref{cap:modeloMacroscopico}, describen el comportamiento dinámico de toda la población de células T, basado en dos características principales  atribuidas a esa población: la elasticidad y la inercia. En ambos casos se realizan simulaciones numéricas de dichos modelos. Estas simulaciones representan distintas situaciones que pueden darse durante una infección. Entre ellas interesa distinguir entre la situación de intolerancia al patógeno, en cuyo caso las células inmunes consiguen controlar la infección y eliminar al agente infeccioso, o la situación de tolerancia al patógeno, en la que es este último quien acaba tomando el control del organismo. También se analiza qué ocurre cuando tenemos poblaciones de células T con distintas afinidades al patógeno. La relación entre las propiedades de ambos modelos constituye un tema interesante, que se aborda en el último capítulo (Capítulo \ref{cap:redNeuronal}). Como veremos, ambos dan lugar a resultados no solo compatibles sino complementarios. 


\section{Objetivos}

Este Trabajo de Fin de Grado se centra en el estudio de la dinámica de población de las células T ante una infección aguda y, más concretamente, en el estudio de dos modelos matemáticos que pretenden dar respuesta a los mecanismos que rigen este comportamiento.

Para abordar este proyecto se determinaron los siguientes objetivos:

 \begin{itemize}
 	
	\item Estudio básico del sistema inmune, enfocado a conocer el papel que juegan las células T durante una \textit{respuesta inmune}.
	
	\item Estudio y comprensión de los modelos matemáticos que se detallan en este documento y de su importancia en el ámbito de la biología.
	
 	\item Implementación del código necesario para simular distintos comportamientos de las células T basados en estos modelos y análisis de dichos comportamientos. 
 	
 	\item Obtención de una primera aproximación para lograr establecer una correspondencia entre los parámetros de los dos modelos estudiados. 
 \end{itemize}


\section{Plan de trabajo}

Para la realización de este trabajo se establecieron distintos hitos a lo largo del curso académico. Los primeros meses estaban destinados a una revisión previa de los conceptos biológicos subyacentes. Los cuales engloban nociones básicas sobre el sistema inmune y un estudio más detallado del comportamiento de las células T. Esto constituye una parte fundamental del trabajo, pues los modelos no pueden ser comprendidos en su totalidad si no se miran desde el problema biológico al que intentan dar respuesta. Una vez se afianzara la base biológica se podía comenzar con el estudio de los modelos. El primero que se estudiaría sería el modelo microscópico expuesto en \cite{JTB}. Se estableció la realización de distintas simulaciones del modelo, que complementarían la teoría vista. El segundo modelo, el modelo macroscópico \citep{arias2015growth}, sería estudiado después, con el propósito de poder relacionarlo con el modelo anterior. 


Una vez ambos modelos estuvieran revisados y se hubieran realizado las simulaciones correspondientes se abrió la posibilidad de intentar establecer una correspondencia entre los parámetros de ambos modelos mediante la implementación de una red neuronal. Esto último pondría fin al contenido de este Trabajo de Fin de Grado.


\section{Estructura del documento}

Este trabajo está dividido en cuatro partes bien diferenciadas, pero con la misma finalidad, el estudio de las células T y su dinámica de población durante una infección aguda. 

\begin{enumerate}
	\item En el Capítulo \ref{cap:estadoDeLaCuestion} se cubre el contexto del documento. En concreto, en la Sección \ref{sec:cuestInmuno} se tratan unas nociones básicas sobre inmunología, que permiten al lector continuar por los capítulos siguientes sin ningún impedimento terminológico, en cuanto a cuestiones biológicas se refiere. Esta sección pretende dar una visión general y muy básica del sistema inmune. Comienza con los mecanismos más simples, referentes al \textit{sistema inmune innato} (Sección \ref{sub:sistInmInnato}), hasta las más complejas, referentes al \textit{sistema inmune adaptativo} (Sección \ref{sub:sistInmAdap}). Más en detalle se exponen los aspectos de la respuesta inmune que involucran a las células T, como son su activación y actuación o la memoria inmune (Sección \ref{Tcell}). 
	
	Por su parte, la Sección \ref{sec:coop} aborda el papel de los modelos matemáticos en el campo de la biología. Concretamente en la Sección \ref{cuestionAmodelizar}, nos centramos en el caso de nuestro estudio, los distintos modelos matemáticos formulados para la dinámica de las células T durante una infección aguda.
	 
	
	\item En el Capítulo \ref{cap:descripcionTrabajo} se expone el marco teórico del modelo microscópico propuesto para el problema de decisión entre división y apóptosis de las células T. En la Sección \ref{sec:hip_bio} se detallan las hipótesis biológicas sobre las que se sustenta el modelo, que constituyen hechos contrastados y observados en el campo de la biología. El modelo en sí puede verse en la Sección \ref{sec:modelo}, donde se detalla la notación que seguirá el resto del documento y las ecuaciones diferenciales de primer orden que dan lugar al algoritmo. La última sección de este capítulo, la Sección \ref{sec:modeloPatCelT}, introduce una ecuación diferencial para la dinámica de población del patógeno y su relación con la cantidad de células T disponibles. La ecuación establece la interacción entre ambas poblaciones.
	
	En el Capítulo \ref{cap:simulaciones} se presentan las simulaciones correspondientes a un caso simplificado del modelo anterior (Sección \ref{sec:modelo_simplif}) y se explican los detalles básicos de la implementación del mismo (Sección \ref{sec:implem_pseudo}). Los resultados de las simulaciones se exponen en la Sección \ref{sec:simulacionesMicro}. Estas simulaciones corresponden a casos de intolerancia y tolerancia al patógeno (Secciones \ref{sim:intoler} y \ref{sim:toler}, respectivamente), así como el caso de la respuesta inmune con poblaciones de células T con distintas afinidades al patógeno (Sección \ref{sim:difPoblacionesT}).
	
	
	\item El modelo macroscópico se estudia en el Capítulo \ref{cap:modeloMacroscopico}. Las ecuaciones diferenciales de segundo grado de este modelo rigen la dinámica de población de las células T y el patógeno de manera colectiva, a diferencia del modelo microscópico, cuyo algoritmo estaba definido para cada una de las células. Este modelo está basado en dos características del comportamiento de la población de células T durante una respuesta inmune: la elasticidad y la inercia (Sección \ref{sec:iner_elast}). 
	
	Además de proponerse un modelo teórico, también se realizan las simulaciones numéricas correspondientes al modelo en la Seccion \ref{sec:simu_macro}. Estas recogen los casos de intolerancia y tolerancia al patógeno (Secciones \ref{sub:simMacroIntoler} y \ref{sub:simMacroToler}, respectivamente) y, para el modelo macroscópico adimensional, se estudia la relevancia del valor de sus dos parámetros en las regiones de tolerancia e intolerancia (Sección \ref{sub:reg_tolerIntolerMacro}).
	
	\item A continuación, una vez estudiados los modelos propuestos en los Capítulos \ref{cap:descripcionTrabajo} y \ref{cap:modeloMacroscopico}, y tras la comparación de sus resultados, se busca una correspondencia de parámetros entre ambos modelos en el Capítulo \ref{cap:redNeuronal}. Para ello se implementa una red neuronal capaz de realizar ``la función inversa'' al código referente a las simulaciones del modelo macroscópico. Es decir, dados los resultados de una simulación, predecir el valor de los parámetros necesarios para obtener ese mismo resultado. La construcción del conjunto de datos y la implementación de la red puede verse en la Sección \ref{sec:conjDatos_entreRed}. Los resultados obtenidos y un ejemplo de ejecución, usando como entrada los resultados de una simulación del modelo microscópico, pueden verse en la Sección \ref{sec:resultadoRed}.
	
\end{enumerate}

Por último, el Capítulo \ref{cap:conclusiones} ofrece una breve conclusión sobre el trabajo realizado. A fin de complementar el proyecto, se ha incluido el código principal de las simulaciones de los modelos (tanto las del Capítulo \ref{cap:simulaciones} como las del \ref{cap:modeloMacroscopico}) en el Apéndice \ref{Appendix:A}.






