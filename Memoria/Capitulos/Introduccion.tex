\chapter{Introducción}
\label{cap:introduccion}

\chapterquote{Frase célebre dicha por alguien inteligente}{Autor}

Las Matemáticas tienen una larga tradición dentro de la Biología, desde los trabajos de Gregor Mendel en genética o los de Theodor Boveri en la naturaleza de los cromosomas. Sin embargo, las colaboraciones matematico-biologo no se hacen notar demasiado frecuentes. Poco a poco los descubrimientos en biología se vuelven más especializados y su entendimiento requiere más detalle, es por eso que los modelos matemáticos que se proponen en este contexto han sido, con frecuencia, mirados ``bajo sospecha''. 

En el caso que nos cocupa, la inmnunología, la cosa no es muy diferente. A pesar de ello, los modelos matemáticos son cada día más importantes. Una de las razones principales es porque la intuición es insuficiente a partir de un cierto nivel de complejidad y el análisis del sistema inmune (SI) debe ser más cuantitativo. 

Los datos recogidos experimentalmente exponen la complejidad del SI, su no linealidad, sus redundancias, etc. Todo esto sumado al avance de la tecnología y la explosión de información, eso que llamamos hoy \textit{big data}, hacen que las soluciones automáticas (computarizadas) sean la única manera de acercarse a determinados problemas biológicos y médicos. 

No debemos olvidar que los modelos matemáticos no son una representación $100\%$ fiable del problema que modelizan, pues la misión que tienen estos modelos es ayudar a comprender el funcionamiento de un determinado proceso cuyo conocimiento aún está incompleto, reproducirlo y predecir qué consecuencias  tendrá. Es, por tanto, importante, remarcar que los modelos se construyen sobre hipótesis aún inestables y que es precisamente esto lo que les hace tan potentes: permiten incluir variaciones, nuevas hipótesis, compararse con otros modelos,... y gracias a ello lograr una visión más amplia del problema. Pudiendo obtener información útil que de otra manera hubiera sido imposible, ya bien sea por razones del elevado coste económico de los experimentos, por el tiempo que lleva realizarlos, o por la cantidad de datos a examinar, entre otras razones. Pero no pensemos que los modelos ``aciertan'', también nos ayudan a descartar vías de investigación que no se ajusten a lo observado. Y eso es, sin duda, avanzar en el problema. 

En este trabajo propondremos un modelo matemático muy simple, basado en ecuaciones diferenciales, con el cual modelizaremos la dinámica de población de unas células del SI muy destacadas: las células T. Además, acompañaremos estos resultados con simulaciones de dicho modelo. 


\section{Motivación}

\begin{itemize}
	\item Matemáticas en este mundillo
	\item (Grandes problemas de la inmunología)
	\item Cómo ayudan las matemáticas
\end{itemize}

\begin{itemize}
	\item Buscar preguntas sin resolver del SI
	\item (Grandes problemas de la inmunología)
\end{itemize}
La habilidad de nuestro sistema inmune (SI) para protegernos de los patógenos es ciertamente apasionante. Las células inmunes deben saber cómo diferenciar a las células amigas de las enemigas, cómo y dónde actuar.

Son diversas las amenazas a las que el SI tiene que enfrentarse y dar una respuesta eficiente y proporcional. Estas amenazas pueden ser de naturaleza biológica (agentes patógenos), físico-químicas (como contaminantes o radiaciones) o internas (por ejemplo, las células cancerosas).

A lo largo de los años, muchas preguntas sobre el funcionamiento del SI han sido respondidas, pero aún quedan muchas otras por responder: ¿Quién regula la actuación del SI? ¿qué influye en la respuesta inmune?, ¿cuál es el \textit{software} que llevan las células inmunes?... PROBLEMAS SIN RESOLVER EN INMUNOLOGÍA EN GENERAL (BUSCAR)

Si bien parece natural pensar que hay un órgano que actúa de director, ese
órgano, si existe, aún no se ha encontrado. Incita, por tanto, a considerar, que las
células inmunes basan su actuación en la información local que encuentran a su
alrededor. Y sobre esta suposición construiremos un modelo que describa las
dos actuaciones básicas, división y muerte celular, que desarrollan las células inmunes que vamos a estudiar: las células T.

De este comportamiento aparentemente complejo destacaremos la simplicidad: las células T tienen un número muy limitado de opciones, y estas vienen determinadas por el ambiente en el que se mueven y la información que recogen de él.


A pesar de que en este trabajo nos centraremos en una tarea muy particular del SI, como es la dinámica de población de las células T, no debemos olvidar que estas células no son las únicas que forman parte de él, hay muchos otras, y de diversos tipos, interactuando con ellas. Pongamos un ejemplo que ayude a entender la dimensión del asunto: supongamos que estamos viendo un partido de fútbol en la televisión y nos enfocan a un jugador que va corriendo a toda velocidad y luego para en seco. Esto no parece tener mucho sentido. Después, repiten la misma jugada con un campo de visión más amplio, donde podemos ver todo el terreno de juego. Ahora entendemos que el jugador ha parado porque el equipo contrario se hizo con el balón que él estaba esperando. 

Para la realización de este estudio presentaremos un modelo matemático, basado en ecuaciones diferenciales, que describa la dinámica de las células T, así como simulaciones de este mismo modelo...(a ver qué se hace aquí).
----------------------

A pesar de la descentralización en sus tareas y del poco margen de maniobra, es asombroso que el resultado que nos ofrece el SI sea tan eficiente.

\section{Objetivos}
 \begin{itemize}
 	\item Estudiar el entorno biológico sobre el que se sustenta este TFG.
 	\item Estudiar y entender el modelo propuesto y sus aplicaciones. 
 	\item Desarrollar simulaciones de dicho modelo que complementen la teoría vista.
 \end{itemize}


\section{Plan de trabajo}
Aquí se describe el plan de trabajo a seguir para la consecución de los objetivos descritos en el apartado anterior.

\section{Estructura del documento}

