\chapter*{Abstract}

%\section*{\tituloPortadaEngVal}
\section*{}

Our immune system is essential to our survival. Without it, we would be defenceless to attacks from bacteria, viruses and parasites. This system acts throughout the body and involves many types of cells. Its main mission is to recognize pathogens and fight them, giving rise to a process called immune response. 


In what follows, we will focus on the population dynamics of a particular type of immune cell: T cells. These cells play a fundamental role in the immune response, as they are active at eliminating those cells in the body that have been infected. The biological mechanisms that determine when and how much these cells reproduce during the immune response have not been fully elucidated yet. 

In this document, two mathematical models, based on differential equations, are presented. They provide a possible explanation to some aspects of immune response from different points of view: one microscopic (at the cellular level) and the other macroscopic (at the level of the entire cell population). We will see that the activity of T cells depends on their decision between cell division or suicide (apoptosis) and that this decision can be studied by means of simple equations, which allow us to formulate the problem in a suitable form for later analysis. In addition to their theoretical framework, several simulations of the models are included. These show their flexible nature, as they allow different situations to be represented simply by changing parameter values in the corresponding equations. This allows for significant information to be inferred without the need for new experiments in a laboratory. Furthermore, a possible correlation between the parameters of the proposed models (microscopic and macroscopic) has been sought in this work.  In particular, and as a first approximation, a neural network has been implemented that allows inferring the parameters of the macroscopic model having as input characteristic aspects of an immune response.


\section*{Keywords}

\noindent mathematical models, differential equations, T cells, pathogens, immune response, numerical simulations.

