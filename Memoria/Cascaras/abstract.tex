\chapter*{Abstract}

%\section*{\tituloPortadaEngVal}
\section*{}

Our immune system is essential to our survival. Without it, we would be exposed to attacks from bacteria, viruses, parasites, and others. This system extends throughout the body and involves many types of cells, organs, proteins and tissues. Its main mission is to recognize pathogens and react to them, giving rise to a process that we call \textit{immune response}. 


In what follows, we will focus on the population dynamics of a particular type of immune cell: T cells. These cells play a fundamental role in the immune response, as they are responsible for eliminating those cells in the body that have been infected. The biological mechanisms that determine when and how much these cells reproduce during the \textit{immune response} have not been elucidated yet. 

That is why, throughout this document, two mathematical models are presented, based on differential equations, which attempt to provide a possible explanation for this phenomenon from different points of view: one microscopic (at the cellular level) and the other macroscopic (at the level of the entire cell population). Despite how complex it may seem, we will see that the decision between T cell division or apoptosis can be modelled by means of simple equations, which allow us to abstract the problem for later analysis. In addition to the theoretical framework of the models, several simulations of the models are included. These show their flexible nature, as they allow different situations to be represented simply by varying the value of their parameters. This allows new knowledge to be inferred without the need for new experiments in a laboratory. Furthermore, in this work, a possible correlation between the parameters of the proposed models has been sought. As a first approximation, a neural network has been implemented that allows inferring the parameters of the macroscopic model having as input characteristic aspects of an \textit{immune response}.


\section*{Keywords}

\noindent T cell, pathogen, immune response, tolerance, intolerance, differential equation, simulation, model, infection, network.



