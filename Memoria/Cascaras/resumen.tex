\chapter*{Resumen}

\section*{\tituloPortadaVal}

Nuestro sistema inmune es esencial para nuestra supervivencia. Sin el, nuestros cuerpos estarían expuestos a ataques de bacterias, virus, parásitos, entre otros. 

Este sistema se extiende por todo el cuerpo e involucra a muchos tipos de células, órganos, proteínas y tejidos. La misión principal de este sistema es reconocer patógenos y reaccionar a ellos, provocando un proceso que llamaremos \textit{respuesta inmune}. 

En lo que sigue nos centraremos en la dinámica de población de un tipo de célula inmune concreto: las células T, estas participan de manera fundamental en la respuesta inmune. A pesar de lo complicado que pueda parecer, veremos que la decisión entre división o muerte de estas células sigue un patrón sumamente sencillo y propondremos un modelo matemático para estas variaciones. Así mismo, se presentarán simulaciones de ejemplo de dicho modelo. 

\section*{Palabras clave}
   
\noindent Máximo 10 palabras clave separadas por comas

   


