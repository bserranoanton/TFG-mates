\chapter*{Resumen}

\section*{\tituloPortadaVal}

Nuestro sistema inmune es esencial para nuestra supervivencia. Sin él estaríamos expuestos a ataques de bacterias, virus, parásitos, entre otros. Este sistema se extiende por todo el cuerpo e involucra a muchos tipos de células, órganos, proteínas y tejidos. Su misión principal es reconocer \textit{patógenos} y reaccionar ante ellos, dando lugar a un proceso que llamamos \textit{respuesta inmune}. 

En lo que sigue nos centraremos en la dinámica de población de un tipo de célula inmune concreto: las células T. Estas participan de manera fundamental en la respuesta inmune, pues son las encargadas de eliminar aquellas células del organismo que han sido infectadas. Los mecanismos biológicos que determinan cuándo y cuánto se reproducen estas células durante la respuesta inmune aún están por resolver. Es por ello que, a lo largo de este documento, se exponen dos modelos matemáticos, basados en ecuaciones diferenciales, que intentan dar una posible explicación a este fenómeno desde puntos de vista distintos: uno microscópico (a nivel celular) y otro macroscópico (a nivel de toda la población de células). A pesar de lo complejo que pueda parecer, veremos que la decisión entre división o apoptosis de las células T puede modelarse por medio de ecuaciones simples, que permiten abstraer el problema para su posterior análisis. Además del marco teórico de los modelos, se incluyen diversas simulaciones de los mismos. En ellos se pone de manifiesto su carácter flexible, pues permiten representar situaciones distintas simplemente con la variación del valor de sus parámetros. Esto permite tanto a biólogos como matemáticos inferir nuevo conocimiento sin necesidad de nuevos experimentos en un laboratorio. Además, en este trabajo se ha buscado una posible correlación entre los parámetros de los modelos propuestos. Como primera aproximación, se ha implementado una red neuronal que permite inferir los parámetros del modelo macroscópico teniendo como entrada aspectos característicos de una \textit{respuesta inmune}.



\section*{Palabras clave}
   
%\noindent Máximo 10 palabras clave separadas por comas

\noindent Célula T, patógeno, inmune, tolerancia, intolerancia, parámetro, simulación, modelo, red.

   


